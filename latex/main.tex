\documentclass[LI,KR]{HSEUniversity}
% Возможные опции: KR или VKR; LI

\title{Применение методов оптимизации нулевого порядка для тонкой настройки мультимодальных больших языковых моделей}
\author{Максим Дмитриевич Татаринов}
\supervisor{доцент факультета информатики, математики и компьютерных наук ВШЭ}{А. В.~Демидовский}
\reviewer{TBD}{П.П.~Петров}

\Year{2025}
\City{Нижний Новгород}

% \Abstract{
% 	Будущая аннотация.
% }

% Ссылка на файл с описание библиографии
\bibliography{library.bib}

%%%%%%%%%%%%%%%%%%%%%%%%%%%%%%%%
%%% ТЕКСТ РАБОТЫ %%%%%%%%%%%%%%%
\begin{document}

% Обязательные элементы оформления: заголовочный слайд, аннотация, оглавление
\maketitle

\chapter*{Введение}


\textbf{Целью} настоящей работы является X.

Из поставленной цели были сформулированы следующие \textbf{задачи}:
\begin{enumerate}
    \item  X.
    \item  X.
    \item  X.
    \item  X.
    \item  X.
    \item  X.
    \item  X.
\end{enumerate}

\textbf{Объектом} исследования является X.

\textbf{Предметом} исследования является X.

Для решения поставленных задач будут использованы следующие \textbf{методы}:
\begin{enumerate}
    \item  X.
    \item  X.
    \item  X.
    \item  X.
    \item  X.
\end{enumerate}

\textbf{Актуальность} настоящей работы состоит в том, что, X.

\textbf{Новизна} настоящей работы состоит в X.

\textbf{Практическая значимость} данной работы заключается в том, что X.


\chapter{X}

\section{X}

X

\putbibliography %Вместо этой команды будет вставлена библиография

\end{document}
